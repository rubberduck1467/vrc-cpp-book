\documentclass[main.tex]{subfiles}
\begin{document}

\chapter{Introduction}

Vex Robotics Competition (VRC) is an international robotics competition for grades 6-12.
Programming is a central part of the competition, allowing the driver finer control of the robot
    and each match has an autonomous portion.
PROS is a 3rd-party API for the VRC Brain with better performance than the native VEX-API.
However, there is very few if any approachable and comprehensive texts on PROS C++,
    so the only way to learn is reading the documentation and reaching out to other programmers.
This text hopes to remedy that by providing a clear and comprehensive reference on programming for VRC.

This text assumes a working knowledge of C++, including variables, functions, header files, and classes,
    and also knowledge of the structure of the VRC Competition.
We will begin with installing the PROS-API and some tools handy for programming, such as Git.
Next, we will cover the structure of a PROS Program and best practices.
Then, we will take a tour of the API and discuss basic robot control.
We will also conduct a simple practical on programming a simple robot.
After that we will cover more advanced algorithms for robot control.
Finally, we will discuss miscellaneous topics such as the Brain LCD, Libraries, and the Judge Interview.
\end{document}
